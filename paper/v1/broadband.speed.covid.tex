% interactcadsample.tex
% v1.03 - April 2017

\documentclass[]{interact}

\usepackage{epstopdf}% To incorporate .eps illustrations using PDFLaTeX, etc.
\usepackage{subfigure}% Support for small, `sub' figures and tables
%\usepackage[nolists,tablesfirst]{endfloat}% To `separate' figures and tables from text if required

\usepackage{natbib}% Citation support using natbib.sty
\bibpunct[, ]{(}{)}{;}{a}{}{,}% Citation support using natbib.sty
\renewcommand\bibfont{\fontsize{10}{12}\selectfont}% Bibliography support using natbib.sty

\theoremstyle{plain}% Theorem-like structures provided by amsthm.sty
\newtheorem{theorem}{Theorem}[section]
\newtheorem{lemma}[theorem]{Lemma}
\newtheorem{corollary}[theorem]{Corollary}
\newtheorem{proposition}[theorem]{Proposition}

\theoremstyle{definition}
\newtheorem{definition}[theorem]{Definition}
\newtheorem{example}[theorem]{Example}

\theoremstyle{remark}
\newtheorem{remark}{Remark}
\newtheorem{notation}{Notation}

% see https://stackoverflow.com/a/47122900


\usepackage{hyperref}
\usepackage[utf8]{inputenc}
\def\tightlist{}

\begin{document}

\articletype{ARTICLE TEMPLATE}

\title{WFH and broadband speed (title needs rework)}


\author{\name{A. N. Author$^{a}$, John Smith$^{b}$}
\affil{$^{a}$Taylor \& Francis, 4 Park Square, Milton Park, Abingdon, UK; $^{b}$Institut für Informatik, Albert-Ludwigs-Universität, Freiburg, Germany}
}

\thanks{CONTACT A. N. Author. Email: \href{mailto:latex.helpdesk@tandf.co.uk}{\nolinkurl{latex.helpdesk@tandf.co.uk}}, John Smith. Email: \href{mailto:john.smith@uni-freiburg.de}{\nolinkurl{john.smith@uni-freiburg.de}}}

\maketitle

\begin{abstract}
TBC
\end{abstract}

\begin{keywords}
covid; internet; working from home; broadband speed; time-series
clusters
\end{keywords}

\hypertarget{introduction}{%
\section{Introduction}\label{introduction}}

During the pandemic, working from home using digital technologies,
whether partially or exclusively, was transformed from a niche means of
accessing work, albeit one that had been on a slow, upward trend, to a
widespread way of life in many countries. The ability to work from home
or telecommute meant millions retained their jobs and, to a varying
extent, maintained productivity during periods of strict lockdown around
the world. However, this ability has not been evenly distributed
socially or spatially, creating a new type of digital divide. On one
side are those who can work from home, supported by digital
technologies, and have thus been able to enjoy both economic resilience
and greater personal safety. On the other side, previously employed
individuals have been forced to accept furlough or redundancy packages
unless they are part of the cadre of essential workers, who are
potentially at high risk of infection. Whilst the basis for this new
digital divide has been viewed as mainly occupational, here we consider
whether the divide is also technological.

Using the UK as a case study, this paper aims to understand how the
quality and reliability of internet service, as reflected in
\emph{experienced} internet speeds, may reinforce or redress the spatial
and social dimensions of the digital division exposed by the pandemic.
To do so, we employ volunteered geographic data on individual broadband
speed tests and state-of-the-art time-series clustering methods to
create clusters of UK local authorities with similar temporal signatures
of experienced internet speeds. We then associate these clusters of
local authorities with their socioeconomic and geographic
characteristics to explore how they overlap with or diverge from the
existing economic and digital geography of the UK. \textbf{need to add
urban-rural to analysis} Our analysis enables us to better understand
how the spatial and social distribution of both occupations and online
accessibility intersect to enable or hinder the practice of
telecommuting at a time of extreme demand. We will also consider what
lessons can be learned from this time for a future where telecommuting
is likely to remain a common means of accessing work, and broadband
services and infrastructure must be fit for purpose. \textbf{LET'S LEAVE
IT FOR NOW, BUT I THINK WE CAN CRYSTALISE MORE THE RQ}

The capability to work from home has previously been studied from the
perspective of whether work tasks in a given occupation both can be and
are allowed to be performed using digital technologies independently of
location or co-location with colleagues, including supervisors
\citep{allen2015effective, singh2013modeling}. However, successful
telecommuting also requires that the quality and reliability of digital
services, particularly home internet connection speeds, enable the
completion of work tasks with a minimum of delay or interruption. Prior
to the pandemic, the performance of broadband services with respect to
telecommuters was never tested at scale, as working from home and
connecting to colleagues and workplace resources via the internet was
the purview of a small minority of workers. Instead, leisure use in the
evening, when video streaming services are at their peak, has been used
to benchmark broadband performance and service delivery by different
Internet Service Providers (ISPs), at least in the UK \citep{ofcom2017}.
Yet the shift towards telecommuting during various stages of lockdown
around the world has been drastic and there are speculations that
post-Covid, the tendency to work from home will be much higher than
pre-Covid, raising questions around whether internet services can
accommodate the increased demand. For example, \(47\)\% of people in
employment in the UK worked soley from home in April \(2020\), whilst
the same figure only reached \(5\)\% the year before
\citep{ons2020, ons2020lm2019}. A back of the envelope calculation
suggests that up to 40\% of the working force could work from home on an
ongoing basis \citep{batty2020editorial}. Similar figures have been
reported for other countries \citep{felstead2020homeworking}. For
instance, \(37\)\% of the European workforce worked from home in April
\(2020\) with countries like Finland reaching \(60\)\%
\citep{eurofound2020}. In the US, almost half of the working population
worked from home during the same period because of the pandemic
\citep{brynjolfsson2020covid}, and a recent estimate indicated that
\(37\)\% of all jobs in the US can be permanently performed entirely
from home \citep{NBERw26948}.

None of these changes could have happened in the absence of reliable
information and communication technology (ICT) infrastructure -- both in
terms of software and hardware. But while software innovations are
easily diffused across space and society\footnote{See for example the
  huge success of videoconferencing apps such as Zoom
  \citep{marks2020zoom}.}, the same does not apply for ICT hardware
infrastructure such as internet broadband connectivity. The literature
describes first level digital divides in terms of the availability and
quality of internet connectivity, such as that manifest in different
geographies in the UK
\citep{riddlesden2014broadband, philip2017digital}. Second level digital
divides consider the presence or lack of the necessary skills to
effectively utilise digital technologies and the internet
\citep{blank2014dimensions, van2011internet}. The third level focuses on
the heterogenous returns of internet usage among different socioeconomic
groups and, consequently, how digital technologies can assist in
bridging or further enhancing existing socioeconomic divides.
\citep{stern2009levels, van2014digital, van2015third}. The capability to
telecommute is related to all three levels of digital divides, but more
importantly leads to differentiated outcomes regarding the economic
resilience of people and places to overcome a systemic shock such as the
current pandemic. We identify this as a new digital divide, one that
fundamentally alters the potential returns of internet use for the user
and wider community, assumes skills or functions that are present in
some occupations but not in others, and relies upon access to high
quality internet services. \textbf{WE NEED TO CONNECT OUR FINDINGS WITH
THIS AND JUSTIFY THAT THIS IS INDEED A NEW DIVIDE} As the quality of
internet infrastructure and services, as well as variation in
occupations are spatially dependent and clustered in space, our approach
offers a framework for understanding which types of places, are more
likely to land on the right side of this new digital divide. By asking
how resilient broadband speeds are as experienced in different parts of
the UK during a time of extreme demand, we interrogate which places
benefit from the greater economic resilience digital technologies can
offer, not only during the pandemic, but also into the future.

The structure of this paper is as follows. First we review the
literature on telecommuting and digital divides to better understand the
origins of the new digital divide revealed by the pandemic and its
impact on the economic resilience of different places. We then describe
our data and methodology. Our results section first offers
classification of how internet services vary across the UK local
authorities and then assesses whether these clusters replicate or
repudiate other socio-economic and geographic patterns of economic
resilience.

\hypertarget{literature-review}{%
\section{Literature review}\label{literature-review}}

\hypertarget{from-telecommuting-to-wfh}{%
\subsection{From telecommuting to
\#WFH}\label{from-telecommuting-to-wfh}}

In this analysis, the terms `telecommuting' and `working from home' are
used interchangeably, as most remote labour during the Covid-19 crisis
was carried out in the homes of individual employees rather than any
other location \citep{eurofound2020}. However, it should be noted that
previous research has explored how telecommuting can occur in other
places, including satellite offices or on public transport
\citep{felstead2012rapid, siha2006telecommuting}. Previous research has
also used a variety of definitions to measure the level of telecommuting
within different workforces, distinguishing, for example, between those
directly employed, indirectly employed, self-employed, full-time or
part-time, and those who use digital technologies to work remotely
full-days or part-days
\citep{allen2015effective, bailey2002review, haddad2009examination}. No
matter the definition, the option and capability to telecommute or work
from home has never been equally distributed spatially or
socio-economically any more than different industries and employment
opportunities have. For example, studies from the United States, the
Netherlands, and the UK indicate that telecommuters are most likely to
hold professional, managerial, and technical occupations, where the
workforce is better educated and wealthier, and that there is suppressed
demand among women and part-time workers
\citep{headicar2016move, peters2004employees, singh2013modeling}.
Opportunities for working from home during the current pandemic have
likewise not been equally spread across the workforce.
\citet{NBERw26948} indicated that in the US, managers, educators, as
well as those working in computer-related occupations, finance, and law
can easily work from home, and that occupations with opportunities to
telecommute are associated with higher earnings. This is not the case
for the workforce occupied in more spatially fixed occupations, from
farming, construction and manufacturing to hospitality and care
services. In the US, these occupations tend to be lower-income,
non-white, without a university degree, live in rental accommodation and
lack health insurance \citep{NBERw27085}. Similar trends can be observed
for other countries. For example, \(75\)\% of workers with tertiary
education worked from home in Europe during spring \(2020\), whilst only
\(34\)\% of workers with secondary education and \(14\)\% of those
primary education did so \citep{eurofound2020}.

\hypertarget{digital-divides-and-economic-resilience}{%
\subsection{Digital divides and economic
resilience}\label{digital-divides-and-economic-resilience}}

Our understanding of telecommuting as a product of enabled occupations
can be described as a manifestation of the third level digital divide,
as those who are able to use digital technologies to work from home
benefit from a high rate of return on their use of the internet in terms
of autonomy, flexibility, and time saved from commuting
\citep{peters2004employees, siha2006telecommuting, singh2013modeling}.
These returns have been even greater during the Covid-19 crisis, when
those with the ability to telecommute also have the ability to maintain
their employment whilst protecting their health. However, the success of
these arrangements has been dependent upon the first level digital
divide, which is associated with access and quality of internet
connectivity. For example, the systematic review from
\citet{SALEMINK2017360} highlights the infrastructure quality
differences between urban and rural areas in various advanced economies.
Whether this variation in infrastructure quality affects the spatial
footprint of telecommuting has not previously been measured, although
there are indications that those who purchase high speed connections
consume more data of all sorts and use their connections for a variety
of purposes \citep{hauge2011consumer}. Another study identifies a
correlation between access to internet services and a reduction in
household transport spend across 33 countries \citep{bris2017ict}.
Whether the implication is less travel because of increased
telecommuting, or whether internet access enables more efficient travel,
this finding is an indication of the potential household savings better
internet services offer. Thus, the extreme demand during the pandemic
provides a new opportunity to understand how infrastructure
accessibility, quality, and reliability affects telecommuting,
particularly as working from home during the pandemic required high
volumes of bandwidth-intensive video conferencing in order to avoid the
face-to-face contact that could increase the spread of infection. By
first answering questions about internet service resilience, we can also
refine our understanding of how this has contributed to or reduced the
new digital divide, where economic resilience has been dependent upon
the capability to work from home.

The multi-layered digital divides intersect with materials divides and
the economic geography of the UK. Following the regional economic
resilience literature, which underlines the differentiated capacity of
cities and regions to escape or recover from economic crises
\citep{martin2012regional, kitsos2018economic}, different places have
different industrial and occupational profiles, and these affect the
aggregated potential capacity of places for telecommuting. Such profiles
are associated with longstanding inequalities in the UK and their
spatial representation as a North-South divide
\citep{martin_north_south}. Various studies have illustrated severe
inequalities between the north and the south part of the UK regarding,
for example, skills and human capital, unemployment, productivity and
prosperity \citep{lee2014grim, mccann2020perceptions, dorling2018peak}.
Some scholars have even argued that the UK suffers some of the highest
level of interregional inequalities in the global north
\citep{gal2018reducing, mccann2016uk}. Not only all three levels of
digital divides are, to a certain extend, associated with or shaped by
the geography of the UK, but the intersection of the digital and
material divides affects the capacity of places to overcome, at least
partially, the economic effects of the Covid-19 pandemic. Importantly,
this is the first time that digital technologies became an essential
tool for economic resilience for such a great part of the population.

\hypertarget{other-subsection}{%
\subsection{other subsection?}\label{other-subsection}}

\textbf{I THINK WE HAVE ENOUGH OF A LIT REV} - circumstances where work
can be carried out more flexibly in space and time - how relate to
digital divides - capability theory of mobility? - broadband tech stuff?
broadband studies / divides / resilience

Some new papers google recommended to me:

\begin{itemize}
\tightlist
\item
  \url{https://urbanstudies.uva.nl/binaries/content/assets/subsites/centre-for-urban-studies/working-paper-series/wps_43.pdf}
\item
  \url{https://link.springer.com/article/10.1007/s11116-020-10136-6}
\item
  \url{https://www.sciencedirect.com/science/article/pii/S0966692319311305}
\item
  check who cites the above and what they cite
\end{itemize}

\textbf{MORE SOURCES:}

\begin{itemize}
\tightlist
\item
  \url{https://www.coronavirusandtheeconomy.com/question/why-has-coronavirus-affected-cities-more-rural-areas}
\item
  \href{https://journals.sagepub.com/toc/EPB/current}{EPB commentaries}
\item
  \url{https://www.coronavirusandtheeconomy.com/question/what-has-coronavirus-taught-us-about-working-home}
\item
  \url{https://www.coronavirusandtheeconomy.com/question/who-can-work-home-and-how-does-it-affect-their-productivity}
\item
  \url{https://www.coronavirusandtheeconomy.com/question/how-will-economic-effects-coronavirus-vary-across-areas-uk}
\item
  \url{https://www.coronavirusandtheeconomy.com/which-parts-uk-have-been-hit-hardest-covid-19-crisis}
\item
  \url{https://www.coronavirusandtheeconomy.com/question/why-has-coronavirus-affected-cities-more-rural-areas}
\end{itemize}

\hypertarget{methods-and-data}{%
\section{Methods and data}\label{methods-and-data}}

The starting point of our methodological framework is cluster analysis,
which can be defined within the modern machine learning framework as an
unsupervised learning task, which involves partitioning unlabelled
observations into homogeneous groups known as clusters
\citep{montero2014tsclust}. The key idea is that observations within
clusters tend to be more similar than observations between clusters.
Clustering is particularly useful for exploratory studies as it
identifies structures within the data \citep{aghabozorgi2015time}.
Therefore, cluster analysis is a widely used family of techniques in
geography \citep{gordon1977classification, everitt1974cluster}. For
instance, clustering methods is the basis of \emph{geodemographics},
which aims to create small area indicators or typologies of
neighbourhoods based on various and some times diverse variables
\citep{SINGLETON2009289, harris2005geodemographics}. Clustering
techniques have also been employed to solve regionalisation problems
\citep{niesterowicz2016}.

Common characteristic of these studies is the cross-sectional nature of
the data they employ. Indeed, most clustering problems in geography deal
with observations that are fixed in time. However, for this paper we are
interested in creating clusters of local authorities in the UK with
similar temporal signatures of experienced internet speeds over time.
Hence, we deviate from the established geographical clustering tools and
employ time-series clustering methods.

Time-series clustering methods have been developed in order to deal with
clustering problems linked to, for instance, stock or other financial
data, economic, governmental or medical data as well as machine
monitoring
\citep{aggarwal2013time, aggarwal2001surprising, hyndman2015large, WARRENLIAO20051857}.
The main challenge -- and also the difference with cross-sectional
clustering problems -- is data dimensionality given the multiplicity of
data points for every individual object -- local authority in our case
-- included in the data set. Time-series are dynamic data as the value
of the observations change as a function of time
\citep{aghabozorgi2015time}. This high dimensionality leads to (i)
computational and algorithmic challenges regarding handling these data
and building algorithms to perform clustering over long time-series, and
(ii) open questions regarding the choice of similarity measures in order
to cluster similar times series objects together considering the whole
length of the time-series and overcoming issues around noise, outliers
and shifts \citep{lin2004iterative, aghabozorgi2015time}.

Time-series clustering methods utilising the whole length of time-series
can be grouped in three categories. The first -- model-based approaches
-- is based on recovering the underlying model for each time-series and
then applying clustering algorithms on the model parameters of each
time-series \citep{aghabozorgi2015time}. The main criticism is the
cluster accuracy for nearby clusters \citep{mitsa2010temporal}. The
second approach is a based the formation of vectors of features based on
the original time-series. These new data of reduced dimensionality is
then clustered using conventional clustering algorithms.

For this paper we utilise the third category of time-series clustering
methods known as shape-based approaches. These methods match two sperate
time-series objects based on the similarity of their shapes through the
calculation of distances among these two shapes and are better equipped
to capture similarities between short length time-series
\citep{aghabozorgi2015time}. This approach serves best this paper
because (i) we aim to identify clusters of UK local authorities with
similar temporal signatures -- i.e.~shapes -- of experienced internet
speeds and (ii) the length of our time-series is short (see the data
discussion in this section).

Another import element of time-series clustering is the actual
clustering algorithm. Similar to clustering of cross-sectional data, we
can employ partitioning algorithms, which lead to non-overlapping
clusters, hierarchical clustering, which result to a hierarchy of
clusters and fuzzy algorithms, which create overlapping clusters
\citep{sardatime}. Because of the simplicity of the implementation and
the interpretability of the results, we utilise here partitioning
clustering based on the widely used \emph{k-Means} algorithm. This is an
iterative algorithm, which begins with defining the desired number of
clusters \emph{k}. Then, each observation is randomly assigned to a
cluster from the \([1,k]\) space. This is the initial cluster
assignment, which is followed by iterations in order to minimise the
distance between the centroids of the clusters and the observations
assigned to these clusters \citep{james2013introduction}.

There are a number of differences between the above described
application of \emph{k-Means} for cross-sectional data and its
application for times series data. Instead of creating clusters around
centroids, a common approach is to create clusters around
\emph{medoids}, which are representative time-series objects with a
minimal distance to all other cluster objects \citep{sardatime}. Also,
instead of calculating the Euclidean distance between centroids and data
points, more complex distance measures need to be employed in order to
capture the similarity between a time-series object and a medoid. A
common distance measure for shape-based time-series clustering in the
Dynamic Time Warping (DTW). Using its underpinning dynamic programming
algorithm, DTW compares two time-series objects to find the optimum
warping path between them. DTW is widely used in order to overcome
limitations linked to the use of Euclidean distance
\citep{sardatime, berndt1994using, ratanamahatana2004everything}. The
\texttt{R} package \texttt{dtwclust} has been used for the time-series
clustering \citep{dtwclust}.

To assess the quality and reliability of internet across local
authorities in the UK during the time when the population were told to
work from home if at all possible we utilise unique data regarding
individual internet speed tests from Speedchecker Ltd\footnote{\url{https://www.broadbandspeedchecker.co.uk/}}.
This a private company that allows internet users to check their own
broadband upload and download speeds, and stores every speed-check with
timestamp and geolocation information. These data have been used before
to assess digital divides \citep{riddlesden2014broadband} and the impact
of local loop unbundling regulatory processes
\citep{nardotto2015unbundling}. These volunteered geographic data enable
to assess the \emph{experienced} internet speed by users, which may
differ from the \emph{advertised} maximum speeds provided by Internet
Service Providers. We are particularly interested in upload speeds and
the frequency of speed tests. While the former is less associated with
internet-based leisure activities such as video streaming, which involve
downloading large amounts of data, the latter can be linked more with
work-related activities such as uploading documents or two-way
communication and interaction systems. \textbf{ADD MORE. HOW ABOUT SPEED
TESTS?}

The first step in the workflow after dropping some outliers following
\citet{riddlesden2014broadband} is to transform and the individual,
geolocated and time-stamped tests to more meaningful aggregates both in
terms of space and time. Regarding the temporal dimension, we aggregate
all the speed-checks during the \(13\) weeks of March to May inclusive
for weekdays in \(2020\) by each hour of the day and day of the week. As
our research aims to identify the geography of internet service
resilience for work purposes, bank holidays and the hours between
midnight and 6am were also excluded. The composite week time-series thus
comprise 18 hours multiplied by 5 weekdays or 90 time points per series.
We aggregate these data because, we could not follow individuals or
households and connect data points. Also, although this is a large data
set (\(241,088\) individual tests during the working hours of study
period), there are not enough observations for each local and for each
working hour (\(631\) speed test per LAD in average). These time-series
were calculated for each of the 382 Local Authority Districts (LADs) in
the UK, standardised, and then a \emph{k}-means partitioning around
medoids clustering algorithm was applied using DTW. We initially run the
algorithm for \(k = 5, 10, 15 and 20\) and used cluster validity indices
(CVIs) to pick the optimal solution of \(k = 10\). Following
\citet{sardatime} the following CVIs were employed: Silhouette (max),
Score function (max), Calinski-Harabasz (max), Davies-Bouldin (min),
Modified Davies-Bouldin (DB*, min), Dunn (max), COP (min).

We then match the cluster membership to the LAD-level data to identify
the characteristics of each cluster, including number of LADs, the
descriptive statistics of upload speeds in that cluster, and the
temporal profile by hour of the day and day of the composite week. Thus
we aim to answer the first part of our research question: How resilient
are broadband speeds as experienced in different parts of the UK during
a time of extreme demand?

Since the quality and reliability of internet services vary in time and
space due to both supply and demand-side influences, we use a number of
different measures of experienced upload speeds. These include: a) mean,
experienced connection speed, b) standard deviation or the amount of
fluctuation from the mean, and c) the variation in speeds at particular
times of day when working from home is more likely to take place. We
take account of all three measurements in order to describe upload
speeds as fully as possible. The following \textbf{graphs / tables} show
these statistics for the sample as a whole, as well as similar
statistics related to frequency of testing.

\textbf{Data details and some figures, descriptive stats - include whole
sample time profile for 2019 and 2020 frequency of tests run as part of
why we chose to create the time profiles by hour of the day and day of
the week rather than daily over the whole period.}

\textbf{Just looking back, it was partly the frequency of tests, but
that was also by geography, so do we need maps? And I feel there was
something else I'm forgetting\ldots{}}

The cause of these different experiences of broadband resilience may be
different in different areas, as they may reflect either similarities in
patterns of demand or similar quality of infrastructure. Our approach is
also limited by potential endogeneity, as for example, better quality
connections with high mean speeds may enable more working from home, but
greater demand may cause slower speeds, less reliability and greater
variability of speed at different times of day or week. Therefore, we
avoid attributing any cause to our analysis of the experienced level of
quality and reliability of upload speeds. Instead, we run an auxiliary
regression in order to understand how the spatial and temporal patterns
of internet service relate to the economic geography of the UK. We
discuss how the different patterns might support or undermine efforts to
work from home and maintain safe productivity and whether they reinforce
existing spatial and social inequalities. More specifically, we estimate
the following multinomial model:

\begin{align}
Pr(Y_{i}=j) = \frac{exp^{X_{i}\beta_{j}}}{\sum_{i=1}^j exp^{X_{i}\beta_{j}}}
\begin{cases}
    i = 1, 2, ... , N \\t
    j = 1, 2, ... , J
\end{cases}\label{eq1}
\end{align}

The identification strategy is as follows. Based on the outcomes of the
time-series clustering, we identify \(J\) distinct and crisp clusters.
We then regress this cluster membership against a vector \(X_{i}\) of
socio-economic and geographic variables, which we are discussed in
details in \textbf{section XX}. This analysis enables us to provide a
more nuanced understanding of how telecommuting and technology intersect
at a time of extreme demand, and what lessons this time has for a future
where telecommuting is likely to remain a common means of accessing work
and broadband services, as well as infrastructure, must be fit for
purpose.

\hypertarget{results}{%
\section{Results}\label{results}}

\hypertarget{upload-clusters-cluster-description}{%
\subsection{Upload Clusters / cluster
description}\label{upload-clusters-cluster-description}}

The temporal profiles used to cluster the local authorities have been
summarised in graph {[}.{]}, which shows a composite profile of mean
upload speeds per hour per day for each cluster. For upload speeds, 345
of 382 local authorities, or over 62 million people, fall into cluster 6
or cluster 9. Graph {[}.{]} suggests that these two clusters have
relatively similar temporal profiles, which are flatter than the other,
smaller clusters, suggesting better reliability of service. However, the
upload speeds at all times for cluster 9 are substantially higher than
for cluster 6, which is an indication of better quality of service. This
difference is reflected not only in the mean speeds for these clusters
for the whole sample, but also the mean upload speeds in the morning
peak from 9:00-10:59, as well as the evening peak period from
19:00-20:59. In comparison, the time profile in graph {[}.{]} shows
upload speeds in cluster 1 are on average lower at certain times of day
during the study period than any other cluster, whilst the profile for
cluster 3 appears to show speeds fluctuating as much as cluster 1, but
at levels usually higher even than cluster 9.

The variability of the smaller clusters may be related to the fewer
speed tests from fewer local authorities that have been averaged, whilst
averaging greater numbers of speed tests could artificially flatten the
profile. This appears not to be the case for cluster 9, as there is
negligible difference between morning and evening upload speeds, at 1\%
slower in the morning, confirming a high level of reliability of
service. In comparison, upload speeds in cluster 6 are 4\% slower in the
morning than in the evening and experienced the joint highest ratio of
standard deviation to mean across the time period under assessment. This
suggests that although the time profile is relatively flat in graph
{[}.{]}, the experience is one of speeds that fluctuate from a lower
mean, and therefore might more often impact on online activities. Still,
in most of the smaller clusters the reliability of service is worse than
in cluster 9 in terms of the ratio of standard deviation to mean. The
smaller clusters are mainly worse than clusters 9 and 6 in terms of the
ratio of upload speed in the morning peak compared to the evening peak.
In five of the smaller clusters, which are home to almost 3 million
people, speeds are 12-19\% slower between 9:00-10:59 compared to
19:00-20:59.

The quality and reliability of broadband service is much better in the
115 Local Authority areas in cluster 9, which are mostly in urban or
suburban areas, compared to cluster 6. These include 13 London Boroughs
(of 32), 8 of the 10 local authorities of Greater Manchester, 5 of the 7
constituent authorities of the West Midlands Combined Authority, as well
as cities like Glasgow, Leicester, Nottingham, Sheffield, and the
Portsmouth and Southampton conurbation. There are also some notable
medium-sized cities, including Aberdeen, Cardiff, Oxford, Milton Keynes,
and York, and many suburban districts from the South East of England to
South Tyneside. Meanwhile, the 230 local authorities in cluster 6, which
have lower speeds on average and more variation in service still include
major urban areas, such as Bristol, Liverpool and Leeds, and many
suburban areas, but also include some of the most rural areas in the
country. Meanwhile, Cluster 1, with 10 local authorities that are home
to over 1 million people has the second slowest speeds in the morning
compared to the evening `peaks' and the second highest ratio of standard
deviation to mean. This cluster's most populous area is Westminster in
central London.

\textbf{further description of temporal profiles of other clusters -
include all in graph?} Indeed, the only cluster where upload speeds were
slower in the evening than in the morning was cluster 5, made up of 2
local authorities with less than 200,000 people: Three Rivers, a
suburban district north of London, and Fylde, a seaside suburb of
Blackpool. However, these are likely to be outliers and may not have
many tests from which the clusters are calcualted. \textbf{Take out
smaller clusters?} The exceptions can be found in seven of the eight
much smaller clusters including 35 local authority districts, where AM
peak upload speeds are between 6\% and 18\% slower than PM peak upload
speeds, although the mean speeds for each cluster are higher than
cluster 6. Indeed, in 25 local authorities with a combined population of
almost 3 million, speeds are 13\% or more slower in the morning than in
the evening. Included in this latter group are central London borough of
Westminster and the London Borough of Newham, rural authorities like
Eden and West Devon, and small cities like Dundee and Carlisle.

\hypertarget{aux-regressions}{%
\subsection{aux regressions}\label{aux-regressions}}

Auxiliary regressions indicate that the speed tests in cluster 9
authorities are also more likely to have been run on services provided
by Virgin Media, suggesting they are in the half of the country with the
most lucrative ICT market, which originally attracted the cable TV
provider (OfCom\ldots). For example, although auxiliary regressions show
that Cluster 6 local authorities are more likely to be in the South of
the UK than Cluster 9, the cluster notably includes Southern rural
districts from Cornwall to North Norfolk. Slower speeds could reflect
the lower quality of service in rural areas compared to urban and
suburban areas. There is frustration, however, as auxiliary regressions
show that those living in cluster 6 had the highest probability of
testing their broadband between 9:00 and 11:00 of any of the 10 upload
speed clusters.

the finding in the auxiliary regressions that those in cluster 9 ran the
fewest speed tests per person during the morning period of any cluster.
Now this may be an indication of fewer people working from home, less
contention, and less resultant frustration. Cluster 9 comprises many
central urban areas and has the lowest number of established businesses
per inhabitant, which could be interpreted as a dominance of large
employers. However, the job density is lower than cluster 6, meaning
there are not as many jobs per resident in these areas. As the cluster
includes many suburban areas too, which may be largely residential,
could the quality and reliability of internet service be reinforcing
patterns of telecommuting by those in wealthier suburbs who can work
from home? Earnings in cluster 9 are second highest of all the clusters,
with only Cluster 2 (comprising just North East Lincolnshire and East
Lothian, population 265k) earning more per person.

The auxiliary regressions show that compared to the two authorities in
cluster 5, all the other clusters had a lower percentage of working
people in managerial, professional and administrative jobs.??? Yet the
auxiliary regression suggests that there are not many tests being run
during the am peak in cluster 1. This may be because there are fewer
people working at home checking their broadband than in most other
clusters. Indeed, the auxiliary regressions indicate that cluster 1 has
the highest job density or proportion of jobs to working-age population,
which is likely to due to the presence of Westminster, central London,
cluster 1's most populous local authority. Westminster not only has more
workplaces than residents, but it is reasonable to presume that many who
would normally work in Westminster, but be able to work from home during
lockdown are likely to live outside central London and not be subject to
the fluctuating speeds there. Workplaces, meanwhile, some of which would
still have been open, could be running programmes that cause the
slowdown and variation, but would be more likely to have their own
in-house diagnostics, rather than using a service like Speedchecker Ltd.

\hypertarget{conclusions}{%
\section{Conclusions}\label{conclusions}}

The long-term effects of such drastic changes in telecommuting and
attitudes towards working from home are difficult to predict.
Nevertheless, they span through various aspects of economy and society:
from changes to transportation planning due to altered commuting
patters, to changes in land use and urban planning to accommodate people
who work from home \citep{BUDNITZ2020102713}\textbf{also 2020 Swedish
article from JTG}; and from productivity and innovation changes, to
changes in agglomeration externalities and the attraction of large
cities \citep{econobs} just to name a few. This paper is positioned to
support endeavours in understanding the effects of increased
telecommuting by exposing the spatial and social dimensions of
telecommuting including the resilience of broadband speeds in terms of
both quality and reliability of service, and whether this reinforces or
redresses prior digital divisions. \textbf{took this bit you wrote to
put in the discussion at the end?}

\hypertarget{acknowledgements}{%
\section*{Acknowledgement(s)}\label{acknowledgements}}
\addcontentsline{toc}{section}{Acknowledgement(s)}

An unnumbered section,
e.g.~\texttt{\textbackslash{}section*\{Acknowledgements\}}, may be used
for thanks, etc.~if required and included \emph{in the non-anonymous
version} before any Notes or References.

\hypertarget{funding}{%
\section*{Funding}\label{funding}}
\addcontentsline{toc}{section}{Funding}

An unnumbered section,
e.g.~\texttt{\textbackslash{}section*\{Funding\}}, may be used for grant
details, etc.~if required and included \emph{in the non-anonymous
version} before any Notes or References.

\bibliographystyle{tfcad}
\bibliography{bibliography.bib}


\input{"appendix.tex"}


\end{document}
