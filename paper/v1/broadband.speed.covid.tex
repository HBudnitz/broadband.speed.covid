% interactcadsample.tex
% v1.03 - April 2017

\documentclass[]{interact}

\usepackage{epstopdf}% To incorporate .eps illustrations using PDFLaTeX, etc.
\usepackage{subfigure}% Support for small, `sub' figures and tables
%\usepackage[nolists,tablesfirst]{endfloat}% To `separate' figures and tables from text if required

\usepackage{natbib}% Citation support using natbib.sty
\bibpunct[, ]{(}{)}{;}{a}{}{,}% Citation support using natbib.sty
\renewcommand\bibfont{\fontsize{10}{12}\selectfont}% Bibliography support using natbib.sty

\theoremstyle{plain}% Theorem-like structures provided by amsthm.sty
\newtheorem{theorem}{Theorem}[section]
\newtheorem{lemma}[theorem]{Lemma}
\newtheorem{corollary}[theorem]{Corollary}
\newtheorem{proposition}[theorem]{Proposition}

\theoremstyle{definition}
\newtheorem{definition}[theorem]{Definition}
\newtheorem{example}[theorem]{Example}

\theoremstyle{remark}
\newtheorem{remark}{Remark}
\newtheorem{notation}{Notation}

% see https://stackoverflow.com/a/47122900


\usepackage{hyperref}
\usepackage[utf8]{inputenc}
\def\tightlist{}

\begin{document}

\articletype{ARTICLE TEMPLATE}

\title{WFH and broadband speed (title needs rework)}


\author{\name{A. N. Author$^{a}$, John Smith$^{b}$}
\affil{$^{a}$Taylor \& Francis, 4 Park Square, Milton Park, Abingdon, UK; $^{b}$Institut für Informatik, Albert-Ludwigs-Universität, Freiburg, Germany}
}

\thanks{CONTACT A. N. Author. Email: \href{mailto:latex.helpdesk@tandf.co.uk}{\nolinkurl{latex.helpdesk@tandf.co.uk}}, John Smith. Email: \href{mailto:john.smith@uni-freiburg.de}{\nolinkurl{john.smith@uni-freiburg.de}}}

\maketitle

\begin{abstract}
TBC
\end{abstract}

\begin{keywords}
covid; internet; working from home; broadband speed; time series
clusters
\end{keywords}

\hypertarget{introduction}{%
\section{Introduction}\label{introduction}}

\href{https://docs.google.com/document/d/1PWjkmgzWGYKR9wFogKYw7l-8mZLoORt593x-Tu-f2-M/edit\#heading=h.i5om1o8wpcd9}{our
Google doc}

During the pandemic, working from home using Information and
Communication Technologies (ICT), whether partially or exclusively, was
transformed from a niche means of accessing work, albeit one that had
been on a slow, upward trend, to a widespread way of life in many
countries. The ability to work from home or telecommute meant millions
retained their jobs and, to a varying extent, maintained productivity
during periods of strict lockdown around the world. However, this
ability was not evenly distributed socially or spatially, creating a new
type of digital divide -- one of economic resilience and personal safety
on one side and unemployment or essential front-line work on the other.
The ability to telecommute is partly a product of whether the majority
of work tasks in an occupation can be performed independently of being
in a particular location or co-locating with colleagues. It is also a
product of the quality of ICT services, particularly home internet
connections, sufficient to successfully complete work tasks with a
minimum of delay or interruption. Therefore, in this paper, we look at
patterns of broadband speed tests and upload speeds in the UK in order
to gain insights into both the quality and reliability of broadband
services across the country and the distribution of telecommuting at the
time when the population were told to work from home if possible. To do
so, we employ unique data regarding individual broadband speed tests and
state of the art time series clustering methods. Our analysis enables us
to better understand how telecommuting and technology intersect at a
time of extreme demand, and what lessons this time has for a future
where telecommuting is likely to remain a common means of accessing work
and broadband services, as well as infrastructure, must be fit for
purpose. \textbf{I THINK WE NEED A MORE CLEAR RESEARCH QUESTION}

The shift towards telecommuting during various stages of lockdown around
the world has been drastic and, importantly, speculations indicate that
the post-Covid tendency to work from home will be much higher than the
pre-Covid one. A back of the envelope calculation suggests that up to
40\% of the working force could work from home in the UK
\citep{batty2020editorial}. Observational data pointed an even higher
share of people in employment in the UK who worked from home in April
\(2020\) (\(47\)\%), while the same figure only reached \(5\)\% the year
before \citep{ons2020, ons2020lm2019}. Similar figures have been
reported for other countries around the world
\citep{felstead2020homeworking}. For instance, \(37\)\% of the workforce
worked from home in Europe in April \(2020\) with countries like Finland
reaching \(60\)\% \citep{eurofound2020}. In the US almost half of the
working population worked from home during the same period because of
the pandemic \citep{brynjolfsson2020covid}. More broadly, a recent
estimate indicated that \(37\)\% of all jobs in the US can be
permanently performed entirely from home \citep{NBERw26948}.

There is a consensus that opportunities for working from home especially
during the current pandemic are not equally spread across the workforce.
\citet{NBERw26948} indicated that in the US managers, educators, as well
as those working in computer-related occupations, finance, and law can
easily work from home. On the contrary, workers in farming,
construction, and manufacturing do not have such opportunities. Not
surprisingly, occupations with opportunities to telecommute are
associated with higher earnings. This is not the case for the workforce
occupied in less footloose occupations as as they tend to be
lower-income, non-white, without a university degree, live in rental
accommodation and lack health insurance \citep{NBERw27085}. Although
these figures refer to the US, similar trends can be observed for other
countries. For example, \(75\)\% of workers with tertiary education
worked from home in Europe during spring \(2020\), while the same share
for workers with secondary and primary education dropped to \(34\)\% and
\(14\)\% respectively. Moreover, employees living in cities, women and
younger employees were have worked from home \citep{eurofound2020}.

None of these changes could have happened in the absence of ICT
infrastructure -- both in terms of software and hardware. But while
innovations in software are easily diffused across space and
society\footnote{See for example the huge success of videoconferencing
  apps \citep{marks2020zoom}.}, the same does not apply for ICT hardware
infrastructure such as internet broadband connectivity. The literature
exemplifies digital divides in terms of internet access and its quality.
For instance, \citet{riddlesden2014broadband} highlighted the broadband
divides in the UK, while the systematic review from
\citet{SALEMINK2017360} reinforced our understanding for the
infrastructure quality differences between urban and rural areas.

Our framework to understand telecommuting as a function of occupations
and quality of ICT infrastructure is aligned with current debates on
digital divides. While the, so-called, first level digital divides are
associated with access and quality of internet connectivity, the second
level ones are linked to the necessary skills to effectively utilise ICT
and the internet \citep{blank2014dimensions, van2011internet}.
Importantly, the differentiated capacity to telecommute, which to a
certain extend are related to the first and second level digital
divides, leads to differentiated outcomes regarding the economic
resilience of people and places against the current pandemic. And to the
extend that the quality of the internet infrastructure and occupation
variation are spatially dependent and clustered in space, the spatial
footprint of telecommuting is of great interest. In a way similar to the
third level digital divide, which focuses on the differentiated returns
of internet use \citep{stern2009levels, van2014digital, van2015third}
places with high rates of telecommuting during the Covid pandemic
illustrate higher economic resilience against the current pandemic. This
is aligned with the regional economic resilience literature, which
underlines the differentiated capacity of cities and regions to escape
or recover from economic crises
\citep{martin2012regional, kitsos2018economic}.

The long-term effects of such drastic changes in telecommuting and
attitudes towards working from home are difficult to predict.
Nevertheless, they might span through various aspects of economy and
society: from changes to transportation planning due to changes in
commuting patters to changes in land use and urban planning to
accommodate people who work from home (\citet{BUDNITZ2020102713}); and
from productivity and innovation changes to a change in agglomeration
externalities and the attraction of large cities \citep{econobs} just to
name a few. This paper is positioned to support such endeavours by
providing an explanatory \textbf{{[}NOTE SURE ABOUT THIS{]}} framework
based on the quality of internet connectivity and the frequencies of
occupations within Local Authorities in the UK.

\textbf{MORE SOURCES:}

\begin{itemize}
\tightlist
\item
  \url{https://www.coronavirusandtheeconomy.com/question/why-has-coronavirus-affected-cities-more-rural-areas}
\item
  \href{https://journals.sagepub.com/toc/EPB/current}{EPB commentaries}
\item
  \url{https://www.coronavirusandtheeconomy.com/question/what-has-coronavirus-taught-us-about-working-home}
\item
  \url{https://www.coronavirusandtheeconomy.com/question/who-can-work-home-and-how-does-it-affect-their-productivity}
\item
  \url{https://www.coronavirusandtheeconomy.com/question/how-will-economic-effects-coronavirus-vary-across-areas-uk}
\item
  \url{https://www.coronavirusandtheeconomy.com/which-parts-uk-have-been-hit-hardest-covid-19-crisis}
\item
  \url{https://www.coronavirusandtheeconomy.com/question/why-has-coronavirus-affected-cities-more-rural-areas}
\end{itemize}

\textbf{PARA5}: Data and methods

\textbf{PARA6}: Contribution:

\hypertarget{literature-review}{%
\section{Literature review}\label{literature-review}}

\hypertarget{broadband-studies-divides-broadband-tech-stuff}{%
\subsection{broadband studies, divides, broadband tech
stuff}\label{broadband-studies-divides-broadband-tech-stuff}}

\hypertarget{from-telecommuting-to-wfh}{%
\subsection{from telecommuting to
\#WFH}\label{from-telecommuting-to-wfh}}

Some new papers google recommended to me:

\begin{itemize}
\tightlist
\item
  \url{https://urbanstudies.uva.nl/binaries/content/assets/subsites/centre-for-urban-studies/working-paper-series/wps_43.pdf}
\item
  \url{https://link.springer.com/article/10.1007/s11116-020-10136-6}
\item
  \url{https://www.sciencedirect.com/science/article/pii/S0966692319311305}
\item
  check who cites the above and what they cite
\end{itemize}

In this analysis, the terms `teleworking' and `working from home' are
used interchangeably. An appropriate definition of teleworking is `the
remote provision of labour that would otherwise be carried out within
company premises' (European Commission, 2020b). In practice, during the
COVID crisis, most such work was carried out in the homes of individual
employees rather than any other location.

\textbf{PARA about Contention}

\hypertarget{time-series-clustering}{%
\section{Time series clustering}\label{time-series-clustering}}

Desription of the method

\hypertarget{data-and-descriptive-statistics}{%
\section{Data and descriptive
statistics}\label{data-and-descriptive-statistics}}

Data details and some figures, descriptive stats

\hypertarget{results}{%
\section{Results}\label{results}}

Clusters, cluster description and aux regressions

\hypertarget{conclusions}{%
\section{Conclusions}\label{conclusions}}

\hypertarget{acknowledgements}{%
\section*{Acknowledgement(s)}\label{acknowledgements}}
\addcontentsline{toc}{section}{Acknowledgement(s)}

An unnumbered section,
e.g.~\texttt{\textbackslash{}section*\{Acknowledgements\}}, may be used
for thanks, etc.~if required and included \emph{in the non-anonymous
version} before any Notes or References.

\hypertarget{funding}{%
\section*{Funding}\label{funding}}
\addcontentsline{toc}{section}{Funding}

An unnumbered section,
e.g.~\texttt{\textbackslash{}section*\{Funding\}}, may be used for grant
details, etc.~if required and included \emph{in the non-anonymous
version} before any Notes or References.

\bibliographystyle{tfcad}
\bibliography{bibliography.bib}


\input{"appendix.tex"}


\end{document}
