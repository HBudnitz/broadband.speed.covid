% interactcadsample.tex
% v1.03 - April 2017

\documentclass[]{interact}

\usepackage{epstopdf}% To incorporate .eps illustrations using PDFLaTeX, etc.
\usepackage{subfigure}% Support for small, `sub' figures and tables
%\usepackage[nolists,tablesfirst]{endfloat}% To `separate' figures and tables from text if required

\usepackage{natbib}% Citation support using natbib.sty
\bibpunct[, ]{(}{)}{;}{a}{}{,}% Citation support using natbib.sty
\renewcommand\bibfont{\fontsize{10}{12}\selectfont}% Bibliography support using natbib.sty

\theoremstyle{plain}% Theorem-like structures provided by amsthm.sty
\newtheorem{theorem}{Theorem}[section]
\newtheorem{lemma}[theorem]{Lemma}
\newtheorem{corollary}[theorem]{Corollary}
\newtheorem{proposition}[theorem]{Proposition}

\theoremstyle{definition}
\newtheorem{definition}[theorem]{Definition}
\newtheorem{example}[theorem]{Example}

\theoremstyle{remark}
\newtheorem{remark}{Remark}
\newtheorem{notation}{Notation}

% see https://stackoverflow.com/a/47122900


\usepackage{hyperref}
\usepackage[utf8]{inputenc}
\def\tightlist{}

\begin{document}

\articletype{ARTICLE TEMPLATE}

\title{WFH and broadband speed (title needs rework)}


\author{\name{A. N. Author$^{a}$, John Smith$^{b}$}
\affil{$^{a}$Taylor \& Francis, 4 Park Square, Milton Park, Abingdon, UK; $^{b}$Institut für Informatik, Albert-Ludwigs-Universität, Freiburg, Germany}
}

\thanks{CONTACT A. N. Author. Email: \href{mailto:latex.helpdesk@tandf.co.uk}{\nolinkurl{latex.helpdesk@tandf.co.uk}}, John Smith. Email: \href{mailto:john.smith@uni-freiburg.de}{\nolinkurl{john.smith@uni-freiburg.de}}}

\maketitle

\begin{abstract}
TBC
\end{abstract}

\begin{keywords}
covid; internet; working from home; broadband speed; time series
clusters
\end{keywords}

\hypertarget{introduction}{%
\section{Introduction}\label{introduction}}

\begin{itemize}
\tightlist
\item
  \href{https://docs.google.com/document/d/1PWjkmgzWGYKR9wFogKYw7l-8mZLoORt593x-Tu-f2-M/edit\#heading=h.i5om1o8wpcd9}{our
  google doc}
\end{itemize}

During the pandemic, working from home using Information and
Communication Technologies (ICT), whether partially or exclusively, was
transformed from a niche means of accessing work, albeit one that had
been on a slow, upward trend, to a widespread way of life in many
countries \citep{felstead2020homeworking}. \textbf{HANNAH, DO WE REALLY
NEED A REFERENCE IN THE PREVIOUS SENTENCE?} The ability to work from
home or telecommute meant millions retained their jobs and, to a varying
extent, maintained productivity during periods of strict lockdown around
the world. However, this ability has not been evenly distributed
socially or spatially, creating a new type of digital divide. On one
side are those who can work from home and have been able to enjoy both
economic resilience and greater personal safety. On the other side,
previously employed individuals have been forced to accept furlough or
redundancy packages unless they are part of the cadre of essential
workers, who are potentially at high risk of infection. Using the UK as
a case study, this paper aims to improve our understanding of the
spatial and social dimensions of this new digital divide by assessing
the resilience of broadband speeds in terms of both quality and
reliability of service, and whether this reinforces or redresses prior
digital divisions. To do so, we employ unique volunteered geographic
data on individual broadband speed tests and state-of-the-art
time-series clustering methods, which enable us to create clusters of UK
local authorities with similar temporal signatures of experienced
internet speeds. We then associate these clusters of local authorities
with their socioeconomic and geographic characteristics to explore
whether they are linked with existing divides and the economic geography
of the UK. \textbf{HANNAH, HOW DOES THIS SOUND?} Our analysis enables us
to better understand how telecommuting and technology intersect at a
time of extreme demand, and what lessons this time has for a future
where telecommuting is likely to remain a common means of accessing work
and broadband services, as well as infrastructure, must be fit for
purpose. \textbf{LET'S LEAVE IT FOR NOW, BUT I THINK WE CAN CRYSTALISE
MORE THE RQ}

The capability to work from home has previously been studied from the
perspective of whether work tasks in a given occupation both can be and
are allowed to be performed independently of location or co-location
with colleagues, including supervisors (\textbf{insert refs}). However,
successful telecommuting also requires that the quality and reliability
of ICT services, particularly home internet connection speeds, enable
the completion of work tasks with a minimum of delay or interruption. In
reality, the performance of broadband speeds with respect to
telecommuters has never been tested before at scale, as working from
home and connecting to colleagues and workplace resources via broadband
has previously been the purview of a small minority of workers. This
papers understands telecommuting as a function of the quality of ICT
services, particularly home internet connections, and whether work tasks
in an occupation can be performed independently of being in a particular
location or co-locating with colleagues

The shift towards telecommuting during various stages of lockdown around
the world has been drastic and, importantly, speculations indicate that
the post-Covid tendency to work from home will be much higher than the
pre-Covid one. A back of the envelope calculation suggests that up to
40\% of the working force could work from home in the UK
\citep{batty2020editorial}. Observational data pointed an even higher
share of people in employment in the UK who worked from home in April
\(2020\) (\(47\)\%), while the same figure only reached \(5\)\% the year
before \citep{ons2020, ons2020lm2019}. Similar figures have been
reported for other countries around the world
\citep{felstead2020homeworking}. For instance, \(37\)\% of the workforce
worked from home in Europe in April \(2020\) with countries like Finland
reaching \(60\)\% \citep{eurofound2020}. In the US almost half of the
working population worked from home during the same period because of
the pandemic \citep{brynjolfsson2020covid}. More broadly, a recent
estimate indicated that \(37\)\% of all jobs in the US can be
permanently performed entirely from home \citep{NBERw26948}.

There is a consensus that opportunities for working from home especially
during the current pandemic are not equally spread across the workforce.
\citet{NBERw26948} indicated that in the US managers, educators, as well
as those working in computer-related occupations, finance, and law can
easily work from home. On the contrary, workers in farming,
construction, and manufacturing do not have such opportunities. Not
surprisingly, occupations with opportunities to telecommute are
associated with higher earnings. This is not the case for the workforce
occupied in less footloose occupations as they tend to be lower-income,
non-white, without a university degree, live in rental accommodation and
lack health insurance \citep{NBERw27085}. Although these figures refer
to the US, similar trends can be observed for other countries. For
example, \(75\)\% of workers with tertiary education worked from home in
Europe during spring \(2020\), while the same share for workers with
secondary and primary education dropped to \(34\)\% and \(14\)\%
respectively. Moreover, employees living in cities, women and younger
employees were have worked from home \citep{eurofound2020}.

None of these changes could have happened in the absence of reliable ICT
infrastructure -- both in terms of software and hardware. But while
software innovations are easily diffused across space and
society\footnote{See for example the huge success of videoconferencing
  apps \citep{marks2020zoom}.}, the same does not apply for ICT hardware
infrastructure such as internet broadband connectivity. The literature
exemplifies digital divides in terms of internet access and its quality.
For instance, \citet{riddlesden2014broadband} highlighted the broadband
divides in the UK, while the systematic review from
\citet{SALEMINK2017360} reinforced our understanding for the
infrastructure quality differences between urban and rural areas.

Our framework to understand telecommuting as a function of occupations
and quality of ICT infrastructure is aligned with current debates on
digital divides. While the, so-called, first level digital divides are
associated with access and quality of internet connectivity, the second
level ones are linked to the necessary skills to effectively utilise ICT
and the internet \citep{blank2014dimensions, van2011internet}.
Importantly, the capacity to telecommute, which to a certain extend is
related to the first and second level digital divides, leads to
differentiated outcomes regarding the economic resilience of people and
places against the current pandemic. And to the extend that the quality
of the internet infrastructure and occupation variation are spatially
dependent and clustered in space, the spatial footprint of telecommuting
is of great interest. In a way similar to the third level digital
divide, which focuses on the differentiated returns of internet use
\citep{stern2009levels, van2014digital, van2015third} places with high
rates of telecommuting during the Covid pandemic illustrate higher
economic resilience against the current pandemic. This is aligned with
the regional economic resilience literature, which underlines the
differentiated capacity of cities and regions to escape or recover from
economic crises \citep{martin2012regional, kitsos2018economic}.

The long-term effects of such drastic changes in telecommuting and
attitudes towards working from home are difficult to predict.
Nevertheless, they span through various aspects of economy and society:
from changes to transportation planning due to altered commuting
patters, to changes in land use and urban planning to accommodate people
who work from home \citep{BUDNITZ2020102713}; and from productivity and
innovation changes, to changes in agglomeration externalities and the
attraction of large cities \citep{econobs} just to name a few. This
paper is positioned to support endeavours in understanding the effects
of increased telecommuting by exposing the spatial and social dimensions
of telecommuting including the resilience of broadband speeds in terms
of both quality and reliability of service, and whether this reinforces
or redresses prior digital divisions. \textbf{HANNAH, HOW DOES THIS
SOUND}

The structure of this paper is as follows. \ldots{}

\textbf{MORE SOURCES:}

\begin{itemize}
\tightlist
\item
  \url{https://www.coronavirusandtheeconomy.com/question/why-has-coronavirus-affected-cities-more-rural-areas}
\item
  \href{https://journals.sagepub.com/toc/EPB/current}{EPB commentaries}
\item
  \url{https://www.coronavirusandtheeconomy.com/question/what-has-coronavirus-taught-us-about-working-home}
\item
  \url{https://www.coronavirusandtheeconomy.com/question/who-can-work-home-and-how-does-it-affect-their-productivity}
\item
  \url{https://www.coronavirusandtheeconomy.com/question/how-will-economic-effects-coronavirus-vary-across-areas-uk}
\item
  \url{https://www.coronavirusandtheeconomy.com/which-parts-uk-have-been-hit-hardest-covid-19-crisis}
\item
  \url{https://www.coronavirusandtheeconomy.com/question/why-has-coronavirus-affected-cities-more-rural-areas}
\end{itemize}

\hypertarget{literature-review}{%
\section{Literature review}\label{literature-review}}

\hypertarget{broadband-studies-divides-broadband-tech-stuff}{%
\subsection{broadband studies, divides, broadband tech
stuff}\label{broadband-studies-divides-broadband-tech-stuff}}

\begin{itemize}
\tightlist
\item
  PARA about Contention??
\end{itemize}

The infrastructural demands of working from home on internet services
have been minor compared to the demands of leisure users, such that the
broadband performance offered by different Internet Service Providers
(ISPs) has been benchmarked according to download speeds during the
evening `primetime', when video streaming services are at their peak,
rather than during the working day (\textbf{OfCom, 2017}). Yet the
pandemic has fundamentally changed not only how many people work from
home, but also their technical requirements when doing so. Whilst it may
have been common in the past for telecommuters to complete solitary work
tasks at home, but still visit their workplace for meetings
(\textbf{ref?}), during the pandemic the replacement of face-to-face
contact with bandwidth-intensive video conference calling was seen as
essential to suppressing the spread of infection. This type of broadband
use can particularly affect upload speeds \textbf{ref?}. This paper
offers a framework to assess the quality and reliability of upload
speeds by time of day and day of the week across local authorities in
the UK during the time when the population were told to work from home
if at all possible.

\hypertarget{from-telecommuting-to-wfh}{%
\subsection{from telecommuting to
\#WFH}\label{from-telecommuting-to-wfh}}

\begin{itemize}
\tightlist
\item
  Introduce space-time geography
\item
  circumstances where work can be carried out more flexibly in space and
  time
\item
  how relate to digital divides
\item
  capability theory of mobility? suppressed demand\ldots{}
\end{itemize}

Some new papers google recommended to me:

\begin{itemize}
\tightlist
\item
  \url{https://urbanstudies.uva.nl/binaries/content/assets/subsites/centre-for-urban-studies/working-paper-series/wps_43.pdf}
\item
  \url{https://link.springer.com/article/10.1007/s11116-020-10136-6}
\item
  \url{https://www.sciencedirect.com/science/article/pii/S0966692319311305}
\item
  check who cites the above and what they cite
\end{itemize}

In this analysis, the terms `teleworking' and `working from home' are
used interchangeably. An appropriate definition of teleworking is `the
remote provision of labour that would otherwise be carried out within
company premises' (European Commission, 2020b). In practice, during the
COVID crisis, most such work was carried out in the homes of individual
employees rather than any other location. \citep{eurofound2020}

\hypertarget{covid-and-working-from-home-cities-urban-structure}{%
\subsection{covid and working from home, cities, urban
structure}\label{covid-and-working-from-home-cities-urban-structure}}

\begin{itemize}
\tightlist
\item
  economic geography of UK
\end{itemize}

\hypertarget{data-and-descriptive-statistics}{%
\section{Data and descriptive
statistics}\label{data-and-descriptive-statistics}}

The experience of upload speeds is not the same as the maximum speeds
offered by an ISP or possible speeds over a particular type and length
of connection. Therefore, we use volunteered geographic information in
the form of speed-checks run by users to test their experienced
broadband speeds, upload and download. Meanwhile, the quality and
reliability of upload speeds, like ICT more generally, vary in time and
space due to both supply and demand-side influences, and can be measured
in a number of ways. These include: a) mean, experienced connection
speed, b) standard deviation or the amount of fluctuation from the mean,
and c) the variation in speeds at particular times of day when working
from home is more likely to take place. We take account of all three
measurements in order to describe upload speeds as fully as possible. We
start by calculating the mean upload speed for every local government
district in the UK for each hour of each weekday, excluding midnight to
6:00, weekends and bank holidays. We then cluster the local authorities
by these temporal profiles, allowing us to identify patterns and
describe the overall means, standard deviations, and other relevant
statistics for each spatial cluster. Thus we aim to answer the first
part of our research question: How resilient are broadband speeds as
experienced in different parts of the UK during a time of extreme
demand?

The cause of these different experiences of broadband resilience may be
different in different areas, as they may reflect similarities in
patterns of demand or similar quality of infrastructure. Our approach is
also limited by potential endogeneity, as for example, better quality
connections with high mean speeds may enable more working from home, but
greater demand may cause slower speeds, less reliability and greater
variability of speed at different times of day or week. Therefore, we
avoid attributing any cause to our analysis of the experienced level of
quality and reliability of upload speeds. Instead, we run auxilliary
regressions in order to understand how the spatial and temporal patterns
of internet service relate to the economic geography of the UK. We
discuss how the different patterns might support or undermine efforts to
work from home and maintain safe productivity and whether they reinforce
existing spatial and social inequalities. From this analysis, we hope to
provide a greater understanding of how telecommuting and technology
intersect at a time of extreme demand, and what lessons this time has
for a future where telecommuting is likely to remain a common means of
accessing work and broadband services, as well as infrastructure, must
be fit for purpose.

The primary data analysed in this paper was provided by Speedchecker
Ltd, a private company that allows internet users to check their own
broadband upload / download speeds, and stores every speed-check with a
timestamp and geolocation.

Our approach involves aggregating all the speed-checks during the 13
weeks of March to May inclusive for weekdays in 2020 by each hour of the
day and day of the week. As our research aims to identify the geography
of internet service resilience for work purposes, bank holidays and the
hours between midnight and 6am were also excluded. The composite week
time series thus comprise 18 hours multiplied by 5 weekdays or 90 time
points per series. These time series were calculated for each of the 382
Local Authority Districts (LADs) in the UK, standardised, and then a
k-means clustering algorithm was applied, including dynamic time
warping. The LADs were assigned to 10 clusters for upload speeds. The
cluster id for each LAD is then reattached to the speed-check dataframe
to identify the characteristics of each cluster, including number of
LADs, and the descriptives statistics of upload speeds in that cluster,
and the temporal profile by hour of the day and day of the week.

\textbf{Data details and some figures, descriptive stats - include whole
sample time profile for 2019 and 2020 frequency of tests run as part of
why we chose to create the time profiles by hour of the day and day of
the week rather than daily over the whole period.}

\hypertarget{time-series-clustering}{%
\section{Time series clustering}\label{time-series-clustering}}

Description of the method

\hypertarget{results}{%
\section{Results}\label{results}}

\hypertarget{upload-clusters-cluster-description}{%
\subsection{Upload Clusters / cluster
description}\label{upload-clusters-cluster-description}}

The temporal profiles used to cluster the local authorities have been
summarised in graph {[}.{]}, which shows a composite profile of mean
upload speeds per hour per day for each cluster. For upload speeds, 345
of 382 local authorities, or over 62 million people, fall into cluster 6
or cluster 9. Graph {[}.{]} shows that both of these clusters have
relatively similar temporal profiles, which are flatter than the other,
smaller clusters, suggesting better reliability of service. However, the
upload speeds at all times for cluster 9 are substantially higher than
for cluster 6, which is an indication of better quality of service. This
difference is reflected not only in the mean speeds for these clusters
for the whole sample, but also the mean upload speeds in the morning
peak from 9:00-10:59, as well as the evening peak period from
19:00-20:59. In comparison, the time profile in graph {[}.{]} shows
upload speeds in cluster 1 are on average lower at certain times of day
during the study period than any other cluster, whilst the profile for
cluster 3 appears to show speeds fluctuating as much as cluster 1, but
at levels usually higher even than cluster 9.

The variability of the smaller clusters may be related to the fewer
speed tests from fewer local authorities that have been averaged, whilst
averaging greater numbers of speed tests could artificially flatten the
profile. This appears not to be the case for cluster 9, as there is
negligible difference between morning and evening upload speeds, at 1\%
slower in the morning, confirming a high level of reliability of
service. In comparison, upload speeds in cluster 6 are 4\% slower in the
morning than in the evening and experienced the joint highest ratio of
standard deviation to mean across the time period under assessment. This
suggests that although the time profile is relatively flat in graph
{[}.{]}, the experience is one of speeds that fluctuate from a lower
mean, and therefore might more often impact on online activities. Still,
in most of the smaller clusters the reliability of service is worse than
in cluster 9 in terms of the ratio of standard deviation to mean and
worse than cluster 6 as well in terms of the ratio of upload speed in
the morning peak compared to the evening peak. In five of the smaller
clusters, which are home to almost 3 million people, speeds are 12-19\%
slower between 9:00-10:59 compared to 19:00-20:59.

The quality and reliability of broadband service is thus much better in
the 115 Local Authority areas in cluster 9, which are mostly in urban or
suburban areas, compared to cluster 6. These include 13 London Boroughs
(of 32), 8 of the 10 local authorities of Greater Manchester, 5 of the 7
constituent authorities of the West Midlands Combined Authority, as well
as cities like Glasgow, Leicester, Nottingham, Sheffield, and the
Portsmouth and Southampton conurbation. There are also some notable
medium-sized cities, including Aberdeen, Cardiff, Oxford, Milton Keynes,
and York, and many suburban districts from the South East of England to
South Tyneside. Meanwhile, the 230 local authorities in cluster 6, which
have lower speeds on average and more variation in service still include
major urban areas, such as Bristol, Liverpool and Leeds, and many
suburban areas, but also include some of the most rural areas in the
country. Meanwhile, Cluster 1, with 10 local authorities that are home
to over 1 million people has the second slowest speeds in the morning
compared to the evening `peaks' and the second highest ratio of standard
deviation to mean. This cluster's most populous area is Westminster in
central London.

\textbf{further description of temporal profiles of other clusters -
include all in graph?} Indeed, the only cluster where upload speeds were
slower in the evening than in the morning was cluster 5, made up of 2
local authorities with less than 200,000 people: Three Rivers, a
suburban district north of London, and Fylde, a seaside suburb of
Blackpool. However, these are likely to be outliers and may not have
many tests from which the clusters are calcualted. \textbf{Take out
smaller clusters?} The exceptions can be found in seven of the eight
much smaller clusters including 35 local authority districts, where AM
peak upload speeds are between 6\% and 18\% slower than PM peak upload
speeds, although the mean speeds for each cluster are higher than
cluster 6. Indeed, in 25 local authorities with a combined population of
almost 3 million, speeds are 13\% or more slower in the morning than in
the evening. Included in this latter group are central London borough of
Westminster and the London Borough of Newham, rural authorities like
Eden and West Devon, and small cities like Dundee and Carlisle.

\hypertarget{aux-regressions}{%
\subsection{aux regressions}\label{aux-regressions}}

Auxiliary regressions indicate that the speed tests in cluster 9
authorities are also more likely to have been run on services provided
by Virgin Media, suggesting they are in the half of the country with the
most lucrative ICT market, which originally attracted the cable TV
provider (OfCom\ldots). For example, although auxiliary regressions show
that Cluster 6 local authorities are more likely to be in the South of
the UK than Cluster 9, the cluster notably includes Southern rural
districts from Cornwall to North Norfolk. Slower speeds could reflect
the lower quality of service in rural areas compared to urban and
suburban areas. There is frustration, however, as auxiliary regressions
show that those living in cluster 6 had the highest probability of
testing their broadband between 9:00 and 11:00 of any of the 10 upload
speed clusters.

the finding in the auxiliary regressions that those in cluster 9 ran the
fewest speed tests per person during the morning period of any cluster.
Now this may be an indication of fewer people working from home, less
contention, and less resultant frustration. Cluster 9 comprises many
central urban areas and has the lowest number of established businesses
per inhabitant, which could be interpreted as a dominance of large
employers. However, the job density is lower than cluster 6, meaning
there are not as many jobs per resident in these areas. As the cluster
includes many suburban areas too, which may be largely residential,
could the quality and reliability of internet service be reinforcing
patterns of telecommuting by those in wealthier suburbs who can work
from home? Earnings in cluster 9 are second highest of all the clusters,
with only Cluster 2 (comprising just North East Lincolnshire and East
Lothian, population 265k) earning more per person.

The auxiliary regressions show that compared to the two authorities in
cluster 5, all the other clusters had a lower percentage of working
people in managerial, professional and administrative jobs.??? Yet the
auxiliary regression suggests that there are not many tests being run
during the am peak in cluster 1. This may be because there are fewer
people working at home checking their broadband than in most other
clusters. Indeed, the auxiliary regressions indicate that cluster 1 has
the highest job density or proportion of jobs to working-age population,
which is likely to due to the presence of Westminster, central London,
cluster 1's most populous local authority. Westminster not only has more
workplaces than residents, but it is reasonable to presume that many who
would normally work in Westminster, but be able to work from home during
lockdown are likely to live outside central London and not be subject to
the fluctuating speeds there. Workplaces, meanwhile, some of which would
still have been open, could be running programmes that cause the
slowdown and variation, but would be more likely to have their own
in-house diagnostics, rather than using a service like Speedchecker Ltd.

\hypertarget{conclusions}{%
\section{Conclusions}\label{conclusions}}

\hypertarget{acknowledgements}{%
\section*{Acknowledgement(s)}\label{acknowledgements}}
\addcontentsline{toc}{section}{Acknowledgement(s)}

An unnumbered section,
e.g.~\texttt{\textbackslash{}section*\{Acknowledgements\}}, may be used
for thanks, etc.~if required and included \emph{in the non-anonymous
version} before any Notes or References.

\hypertarget{funding}{%
\section*{Funding}\label{funding}}
\addcontentsline{toc}{section}{Funding}

An unnumbered section,
e.g.~\texttt{\textbackslash{}section*\{Funding\}}, may be used for grant
details, etc.~if required and included \emph{in the non-anonymous
version} before any Notes or References.

\bibliographystyle{tfcad}
\bibliography{bibliography.bib}


\input{"appendix.tex"}


\end{document}
